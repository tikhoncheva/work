\documentclass[
	fontsize=12pt,
	paper=a4,
	twoside=false,
	numbers=noenddot,
	plainheadsepline,
	toc=listof,
	toc=bibliography
]{scrartcl}

\usepackage[german,ngerman]{babel} % Silbentrennung
%\usepackage[T1]{fontenc} % Ligaturen, richtige Umlaute im PDF 
\usepackage[utf8]{inputenc}% UTF8-Kodierung für Umlaute usw

\usepackage[round]{natbib}

\usepackage{amssymb,amsmath}

\usepackage{placeins}
\usepackage{float}
\restylefloat{table}
\restylefloat{figure}

\usepackage{tikz}


\usepackage{hyperref}

% Literaturverzeichnis
\usepackage{natbib}

% Absätze
\setlength{\parindent}{0pt}

\usepackage{graphicx}

\begin{document}
\section*{Ergebnisse der Berechnungen}

% -------------------------------------------------------------------------------
\subsection*{Modelle}

Wir betrachten zur erst das klassische Modell zum Frequenzzuordnungsproblem (engl. Channel Assigment Problem, CAP). Dieses Problem ist dem $L(d_1, d_2, \dots, d_r)$-Labeling-Problem der Graphentheorie äquivalent.

Gegeben seien ein zusammenhängender Graph $G=(V,\ E)$ und eine Folge $d_1\ge d_2\ge\dots\ge d_r$ der ganzen Zahlen, wobei $r$ kleine gleich maximales Durchmesser des Graphen $G$ ist. Wir bezeichnen mit $dist(i,j)$ die Distanz zwischen den Knoten $i,j\in V$. Gesucht wird eine minimale Zahl $\lambda$, sodass Graph $G$ eine $L(d_1, d_2, \dots, d_r)$-Labeling $c$ mit $\lambda = \max\{c(i), i\in V\}$ hat.




Das von uns betrachtete Mixed-Integer-Optimierungsproblem sieht folgendermaßen aus: 
\begin{align}\label{model}
& \min_{\lambda\in\mathbb{Z}}\lambda & \\
\label{m1} & \text{\bf s.t.}\ c_{i}\le\lambda,& \forall i\in V \\
\label{m2} & \hspace{23 pt} c_{i}\ge 0, & \forall i\in V \\
\label{m3} & \hspace{23 pt} c_j-c_i + M z_{ij}\ge d_{dist(i,j)}, & \forall i,j\in V\\
\label{m4} & \hspace{23 pt} c_i-c_j + M(1-z_{ij})\ge d_{dist(i,j)}& \forall i,j\in V\\
\label{m5} & \hspace{23 pt} z_{ij}\in \{0,1\}& \forall i,j\in V \\ 
\label{m6} & \hspace{23 pt} c_{i}\in \mathbb{Z},& \forall i\in V
\end{align}

Die zusätzliche Variable $z_{ij}$ ist gleich $1$, falls $c_{i}\ge c_{j}$ gilt, und $0$ sonst. Daraus folgt, dass die Bedingungen \ref{m3} und \ref{m4} mit dem hinreichend großen $M$ der Bedingung $\lvert c_i-c_j\rvert\ge d_{dist(i,j)}$ entsprechen.

Wir möchten nun dieses Problem auf den mehr allgemeineren Fall der reelwertigen Variablen $c_i, i\in\mathbb{R},$ übertragen.
Dafür formulieren wir ein erweitertes Model:

\begin{align}\label{model2}
& \min_{\lambda\in\mathbb{R}}\lambda & \\
\label{m21} & \text{\bf s.t.}\ c_{i}\le\lambda,& \forall i\in V \\
\label{m22} & \hspace{23 pt} c_{i}\ge 0, & \forall i\in V \\
\label{m23} & \hspace{23 pt} c_j-c_i + M z_{ij}\ge f({dist(i,j)}), & \forall i,j\in V\\
\label{m24} & \hspace{23 pt} c_i-c_j + M(1-z_{ij})\ge f({dist(i,j)})& \forall i,j\in V\\
\label{m25} & \hspace{23 pt} z_{ij}\in \{0,1\}& \forall i,j\in V \\
\label{m26} & \hspace{23 pt} c_{i}\in \mathbb{R},& \forall i\in V
\end{align}
Hier unter der Distanz $dist(i,j)$wird  eine euklidische Distanz zwischen den Knoten $i,j$ gemeint und Funktion $f$ ist eine Abbildung $\mathbb{R}\rightarrow\mathbb{R}$.

Wir möchten die zwei formulierte Modelle vergleichen. Dafür betrachten wir im klassischen Fall die $L(2,1)$ und $L(3,2,1)$-Labelings. Dies entspricht der Wahl der linearen Funktion $f(x)= d-x$ mit den Konstanten $d=3$ und $d=4$ im zweiten Fall.

% -------------------------------------------------------------------------------
\subsection*{Gridgraphen}
Der Vergleich wird für spezielle Graphen durchgeführt, nämlich Gridgraphen. Dabei werden die Graphen mit den Zellen als Dreiecke (insgesamt $23$ Knoten), Vierecke (insgesamt $25$ Knoten) und Sechsecke betrachtet (insgesamt $24$ Knoten). 

\begin{figure}[hb]
\begin{minipage}[h]{0.32\linewidth}
	\centering
    \vbox{ \begin{tikzpicture}
		\node[label=above:$\bf{0}$] (v0) at (0,0) {};\fill (v0) circle (2pt);
		\node[label=above:$\bf{1}$] (v1) at (1,0) {};\fill (v1) circle (2pt);
		\node[label=above:$\bf{2}$] (v2) at (2,0) {};\fill (v2) circle (2pt);
		\node[label=above:$\bf{3}$] (v3) at (3,0) {};\fill (v3) circle (2pt);
		\node[label=above:$\bf{4}$] (v4) at (4,0) {};\fill (v4) circle (2pt);
		
		\node[label=above:$\bf{5}$] (v5) at (0,-1) {};\fill (v5) circle (2pt);
		\node[label=above:$\bf{6}$] (v6) at (1,-1) {};\fill (v6) circle (2pt);
		\node[label=above:$\bf{7}$] (v7) at (2,-1) {};\fill (v7) circle (2pt);
		\node[label=above:$\bf{8}$] (v8) at (3,-1) {};\fill (v8) circle (2pt);
		\node[label=above:$\bf{9}$] (v9) at (4,-1) {};\fill (v9) circle (2pt);
	
		\node[label=above:$\bf{10}$] (v10) at (0,-2) {};\fill (v10) circle (2pt);
		\node[label=above:$\bf{11}$] (v11) at (1,-2) {};\fill (v11) circle (2pt);
		\node[label=above:$\bf{12}$] (v12) at (2,-2) {};\fill (v12) circle (2pt);
		\node[label=above:$\bf{13}$] (v13) at (3,-2) {};\fill (v13) circle (2pt);
		\node[label=above:$\bf{14}$] (v14) at (4,-2) {};\fill (v14) circle (2pt);
	
		\node[label=above:$\bf{15}$] (v15) at (0,-3) {};\fill (v15) circle (2pt);
		\node[label=above:$\bf{16}$] (v16) at (1,-3) {};\fill (v16) circle (2pt);
		\node[label=above:$\bf{17}$] (v17) at (2,-3) {};\fill (v17) circle (2pt);
		\node[label=above:$\bf{18}$] (v18) at (3,-3) {};\fill (v18) circle (2pt);
		\node[label=above:$\bf{19}$] (v19) at (4,-3) {};\fill (v19) circle (2pt);
	
		\node[label=above:$\bf{20}$] (v20) at (0,-4) {};\fill (v20) circle (2pt);
		\node[label=above:$\bf{21}$] (v21) at (1,-4) {};\fill (v21) circle (2pt);
		\node[label=above:$\bf{22}$] (v22) at (2,-4) {};\fill (v22) circle (2pt);
		\node[label=above:$\bf{23}$] (v23) at (3,-4) {};\fill (v23) circle (2pt);
		\node[label=above:$\bf{24}$] (v24) at (4,-4) {};\fill (v24) circle (2pt);
		
		\draw[-] (v0) to (v1); \draw[-]  (v0) to (v5);
		\draw[-] (v1) to (v2); \draw[-]  (v1) to (v6);
		\draw[-] (v2) to (v3); \draw[-]  (v2) to (v7);
		\draw[-] (v3) to (v4); \draw[-]  (v3) to (v8);
		\draw[-] (v4) to (v9);
		\draw[-] (v5) to (v6); \draw[-]  (v5) to (v10);
		\draw[-] (v6) to (v7); \draw[-]  (v6) to (v11);
		\draw[-] (v7) to (v8); \draw[-]  (v7) to (v12);
		\draw[-] (v8) to (v9); \draw[-]  (v8) to (v13);
		\draw[-] (v9) to (v14);
		\draw[-] (v10) to (v11); \draw[-]  (v10) to (v15);
		\draw[-] (v11) to (v12); \draw[-]  (v11) to (v16);
		\draw[-] (v12) to (v13); \draw[-]  (v12) to (v17);
		\draw[-] (v13) to (v14); \draw[-]  (v13) to (v18);
		\draw[-] (v14) to (v19);
	
		\draw[-] (v15) to (v16); \draw[-]  (v15) to (v20);
		\draw[-] (v16) to (v17); \draw[-]  (v16) to (v21);
		\draw[-] (v17) to (v18); \draw[-]  (v17) to (v22);
		\draw[-] (v18) to (v19); \draw[-]  (v18) to (v23);
		\draw[-] (v19) to (v24);
		\draw[-] (v20) to (v21);
		\draw[-] (v21) to (v22);
		\draw[-] (v22) to (v23);
		\draw[-] (v23) to (v24);
    \end{tikzpicture}}
\end{minipage}
\hfill
\begin{minipage}[h]{0.32\linewidth}
	\centering
    \vbox{ \begin{tikzpicture}
		\node[label=above:$\bf{0}$] (v0) at (0,0) {};\fill (v0) circle (2pt);
		\node[label=above:$\bf{1}$] (v1) at (1.73,0) {};\fill (v1) circle (2pt);

		\node[label=above:$\bf{2}$] (v2) at (-0.87,-0.5) {};\fill (v2) circle (2pt);
		\node[label=above:$\bf{3}$] (v3) at (0.87,-0.5) {};\fill (v3) circle (2pt);
		\node[label=above:$\bf{4}$] (v4) at (2.6,-0.5) {};\fill (v4) circle (2pt);
		
		\node[label=above:$\bf{5}$] (v5) at (-0.87,-1.5) {};\fill (v5) circle (2pt);
		\node[label=above:$\bf{6}$] (v6) at (0.87,-1.5) {};\fill (v6) circle (2pt);
		\node[label=above:$\bf{7}$] (v7) at (2.6,-1.5) {};\fill (v7) circle (2pt);

		\node[label=above:$\bf{8}$] (v8) at (-1.73,-2) {};\fill (v8) circle (2pt);
		\node[label=above:$\bf{9}$] (v9) at (0,-2) {};\fill (v9) circle (2pt);
		\node[label=above:$\bf{10}$] (v10) at (1.73,-2) {};\fill (v10) circle (2pt);
		\node[label=above:$\bf{11}$] (v11) at (3.46,-2) {};\fill (v11) circle (2pt);

		\node[label=above:$\bf{12}$] (v12) at (-1.73,-3) {};\fill (v12) circle (2pt);
		\node[label=above:$\bf{13}$] (v13) at (0,-3) {};\fill (v13) circle (2pt);
		\node[label=above:$\bf{14}$] (v14) at (1.73,-3) {};\fill (v14) circle (2pt);
		\node[label=above:$\bf{15}$] (v15) at (3.46,-3) {};\fill (v15) circle (2pt);

		\node[label=above:$\bf{16}$] (v16) at (-0.87,-3.5) {};\fill (v16) circle (2pt);
		\node[label=above:$\bf{17}$] (v17) at (0.87,-3.5) {};\fill (v17) circle (2pt);
		\node[label=above:$\bf{18}$] (v18) at (2.6,-3.5) {};\fill (v18) circle (2pt);

		\node[label=above:$\bf{19}$] (v19) at (-0.87,-4.5) {};\fill (v19) circle (2pt);
		\node[label=above:$\bf{20}$] (v20) at (0.87,-4.5) {};\fill (v20) circle (2pt);
		\node[label=above:$\bf{21}$] (v21) at (2.6,-4.5) {};\fill (v21) circle (2pt);

		\node[label=above:$\bf{22}$] (v22) at (0,-5) {};\fill (v22) circle (2pt);
		\node[label=above:$\bf{23}$] (v23) at (1.73,-5) {};\fill (v23) circle (2pt);

		
		\draw[-] (v0) to (v2); \draw[-]  (v0) to (v3);
		\draw[-] (v1) to (v3); \draw[-]  (v1) to (v4);
		\draw[-] (v2) to (v5); 
		\draw[-] (v3) to (v6);
		\draw[-] (v4) to (v7);
		\draw[-] (v5) to (v8); \draw[-] (v5) to (v9);
		\draw[-] (v6) to (v9); \draw[-] (v6) to (v10);
		\draw[-] (v7) to (v10);\draw[-] (v7) to (v11);
		\draw[-] (v8) to (v12);
		\draw[-] (v9) to (v13);
		\draw[-] (v10) to (v14);
		\draw[-] (v11) to (v15);
		\draw[-] (v12) to (v16);
		\draw[-] (v13) to (v16); \draw[-] (v13) to (v17);
		\draw[-] (v14) to (v17); \draw[-] (v14) to (v18);
		\draw[-] (v15) to (v18);
		\draw[-] (v16) to (v19);
		\draw[-] (v17) to (v20);
		\draw[-] (v18) to (v21);
		\draw[-] (v19) to (v22);
		\draw[-] (v20) to (v22); \draw[-] (v20) to (v23);
		\draw[-] (v21) to (v23);
    \end{tikzpicture}}
\end{minipage}
\hfill
\begin{minipage}[h]{0.32\linewidth}
	\centering
    \vbox{ \begin{tikzpicture}
		\node[label=above:$\bf{0}$] (v0) at (0,0) {};\fill (v0) circle (2pt);
		\node[label=above:$\bf{1}$] (v1) at (1,0) {};\fill (v1) circle (2pt);
		\node[label=above:$\bf{2}$] (v2) at (2,0) {};\fill (v2) circle (2pt);
		\node[label=above:$\bf{3}$] (v3) at (3,0) {};\fill (v3) circle (2pt);
		\node[label=above:$\bf{4}$] (v4) at (4,0) {};\fill (v4) circle (2pt);
		
		\node[label=above:$\bf{5}$] (v5) at (0.5,-1) {};\fill (v5) circle (2pt);
		\node[label=above:$\bf{6}$] (v6) at (1.5,-1) {};\fill (v6) circle (2pt);
		\node[label=above:$\bf{7}$] (v7) at (2.5,-1) {};\fill (v7) circle (2pt);
		\node[label=above:$\bf{8}$] (v8) at (3.5,-1) {};\fill (v8) circle (2pt);
	
		\node[label=above:$\bf{9}$] (v9) at (0,-2) {};\fill (v9) circle (2pt);
		\node[label=above:$\bf{10}$] (v10) at (1,-2) {};\fill (v10) circle (2pt);
		\node[label=above:$\bf{11}$] (v11) at (2,-2) {};\fill (v11) circle (2pt);
		\node[label=above:$\bf{12}$] (v12) at (3,-2) {};\fill (v12) circle (2pt);
		\node[label=above:$\bf{13}$] (v13) at (4,-2) {};\fill (v13) circle (2pt);
	
		\node[label=above:$\bf{14}$] (v14) at (0.5,-3) {};\fill (v14) circle (2pt);
		\node[label=above:$\bf{15}$] (v15) at (1.5,-3) {};\fill (v15) circle (2pt);
		\node[label=above:$\bf{16}$] (v16) at (2.5,-3) {};\fill (v16) circle (2pt);
		\node[label=above:$\bf{17}$] (v17) at (3.5,-3) {};\fill (v17) circle (2pt);

	
		\node[label=above:$\bf{18}$] (v18) at (0,-4) {};\fill (v18) circle (2pt);
		\node[label=above:$\bf{19}$] (v19) at (1,-4) {};\fill (v19) circle (2pt);
		\node[label=above:$\bf{20}$] (v20) at (2,-4) {};\fill (v20) circle (2pt);
		\node[label=above:$\bf{21}$] (v21) at (3,-4) {};\fill (v21) circle (2pt);
		\node[label=above:$\bf{22}$] (v22) at (4,-4) {};\fill (v22) circle (2pt);
		
		\draw[-] (v0) to (v1); \draw[-]  (v0) to (v5);
		\draw[-] (v1) to (v2); \draw[-]  (v1) to (v5); \draw[-]  (v1) to (v6);
		\draw[-] (v2) to (v3); \draw[-]  (v2) to (v6); \draw[-]  (v2) to (v7);
		\draw[-] (v3) to (v4); \draw[-]  (v3) to (v7); \draw[-]  (v3) to (v8);
		\draw[-] (v4) to (v8);
		\draw[-] (v5) to (v6); \draw[-]  (v5) to (v9); \draw[-]  (v5) to (v10);
		\draw[-] (v6) to (v7); \draw[-]  (v6) to (v10); \draw[-]  (v6) to (v11);
		\draw[-] (v7) to (v8); \draw[-]  (v7) to (v11); \draw[-]  (v7) to (v12);
		\draw[-] (v8) to (v12); \draw[-]  (v8) to (v13);
		\draw[-] (v9) to (v10); \draw[-] (v9) to (v14);
		\draw[-] (v10) to (v11); \draw[-]  (v10) to (v14);\draw[-]  (v10) to (v15);
		\draw[-] (v11) to (v12); \draw[-]  (v11) to (v15);\draw[-]  (v11) to (v16);
		\draw[-] (v12) to (v13); \draw[-]  (v12) to (v16);\draw[-]  (v12) to (v17);
		\draw[-]  (v13) to (v17);
		\draw[-] (v14) to (v15); \draw[-] (v14) to (v18); \draw[-] (v14) to (v19);
		\draw[-] (v15) to (v16); \draw[-] (v15) to (v19); \draw[-] (v15) to (v20);
		\draw[-] (v16) to (v17); \draw[-] (v16) to (v20); \draw[-] (v16) to (v21);
		\draw[-] (v17) to (v21); \draw[-] (v17) to (v22);
		\draw[-] (v18) to (v19);
		\draw[-] (v19) to (v20);
		\draw[-] (v20) to (v21);
		\draw[-] (v21) to (v22);
    \end{tikzpicture}}
\end{minipage}
\begin{minipage}[ht]{1\linewidth}
\begin{tabular}{p{0.32\linewidth}p{0.32\linewidth}p{0.32\linewidth}}
\centering a) Square Lattice & \centering b) Hexagonal Lattice  & \centering c) Triangular Lattice \\
\end{tabular}
\end{minipage}
\vspace*{-1cm}
\caption{ In Betracht genommene Gridgraphen}
\label{Abb.1}
\end{figure}


Für diese Graphen lassen sich die Werte der $L(2,1)$ und $L(3,2,1)$-Labelings einfach berechnen. Das Modell mit den reellen Zahlen ist schwerer zu lösen und braucht deswegen deutlich mehr Zeit (siehe Tabelle \ref{Table:Table1}). 

\begin{table}[htbp]
\centering
  \begin{tabular}{|c|c|c| c|c|c|}
  \hline Lattice & $f(x) =3-x$  & $L(2,1)$ & & $f(x) =4-x$  & $L(3,2,1)$\\ \hline 
    Hexagonal & $6.41699$ & $5$ & &  $16.6243$ & $9$ \\ 
			& $1.69$ sec & $0.08$ sec& & $1838.73$ sec	& $0.29$ sec\\ \hline
	Triangular & $9.78029$ & $8$ & & $21.9068$ & $18$ \\
			& $10.96$ sec	& $0.34$ sec& & $103883$ sec & $167.29$ sec\\ \hline
    Square	& $8.63494$ & $6$ & & $19.9067$ & $11$ \\
			& $4.92$ sec & $0.2$ sec & & $27793.8$ sec & $0.68$ sec\\
  \hline
  \end{tabular}
\caption{Ergebnisse für  $L(2,1)$,$L(3,2,1)$ im klassischen Fall und Funktion $f(x)=3-x$, $f(x)=4-x$ im Fall der reelwertiger Labeling }
\label{Table:Table1}
\end{table}

% -------------------------------------------------------------------------------
\subsection{Treppenfunktion}

Alternativ zu der linear absteigender Funktion kann für $f$ eine Treppenfunktion gewählt werden. Anhand der Zusammenhang zwischen den Graphdistanzen und euklidischen Distanzen,
siehe Table \ref{TreppFunk1}, \ref{TreppFunk2}, \ref{TreppFunk3}, haben wir die Treppenfunktionen für die entsprechende klassische Formulierung wie folgt definiert:
\[ f_{hexagonal,L(2,1)}(x) = \left\{
  \begin{array}{l l}
    2 & \quad \text{if $x\le 1$}\\
    1 & \quad \text{if $x\le \sqrt{3}$}
  \end{array} \right.\]

\[ f_{hexagonal,L(3,2,1)}(x) = \left\{
  \begin{array}{l l}
    3 & \quad \text{if $x\le 1$}\\
    2 & \quad \text{if $x\le \sqrt{3}$}\\
    1 & \quad \text{if $x\le \sqrt{7}$}
  \end{array} \right.\]


\[ f_{triangular,L(2,1)}(x) = \left\{
  \begin{array}{l l}
    2 & \quad \text{if $x\le 1$}\\
    1 & \quad \text{if $x\le 2$}
  \end{array} \right.\]

\[ f_{triangular,L(3,2,1)}(x) = \left\{
  \begin{array}{l l}
    3 & \quad \text{if $x\le 1$}\\
    2 & \quad \text{if $x\le 2$}\\
    1 & \quad \text{if $x\le 3$}
  \end{array} \right.\]

\[ f_{square,L(2,1)}(x) = \left\{
  \begin{array}{l l}
    2 & \quad \text{if $x\le 1$}\\
    1 & \quad \text{if $x\le 2$}
  \end{array} \right.\]

\[ f_{square,L(3,2,1)}(x) = \left\{
  \begin{array}{l l}
    3 & \quad \text{if $x\le 1$}\\
    2 & \quad \text{if $x\le 2$}\\
    1 & \quad \text{if $x\le 3$ und $x\neq\sqrt{8}$}
  \end{array} \right.\]

Die Ergebnisse der Berechnungen mit den Treppenfunktionen sind gleich den aus der klassischen Formulierung, brauchen aber mehr Zeit, um fertig zu werden (vgl. Table \ref{Table3}).

\begin{table}[htbp]
\centering
  \begin{tabular}{|c|c|c|c|c|c|}
    \hline
    Lattice& $f(x)$  & $L(2,1)$ & & f(x) & $L(3,2,1)$ \\ \hline
    Hexagonal	& $5$ & $5$ & & $9$ & $9$ \\
			& $0.61$ sec	& $0.08$ sec & & $1.59$ & $0.29$ sec\\ \hline
    Triangular	& $8$ & $8$ & & $18$ &  $18$\\
			& $7.94$ sec & $0.34$ sec & & $1170.4$ sec & $167.29$ sec \\ \hline
    Square	& $6$ & $6$ & & $11$ & $11$\\
			& $6.51$ sec & $0.2$ sec & & $68.91$ sec & $0.68$ sec\\   \hline
  \end{tabular}
\caption{Ergebnisse für $L(2,1)$ und $L(3,2,1)$ im klassischen Fall und Treppenfunktionen im Fall der reelwertiger Labeling}
\label{Table3}
\end{table}

\begin{table}[tb]
\centering
  \begin{tabular}{|c|c|}
  \hline
   Graphdistanz	& euklidische Distanz \\ \hline
		1	& $1$\\
		2	& $\sqrt{3}$\\
		3	& $2$,$\sqrt{7}$\\
		4	& $3$, $\sqrt{12}$\\
		5	& $3\sqrt{13}$, $4$, $\sqrt{19}$\\
		6	& $\sqrt{21}$, $\sqrt{27}$\\
		7	& $5$, $\sqrt{28}$\\
  \hline
  \end{tabular}
\caption{Graph- und euklidische Distanzen in einem hexagonalem Gridgraphen mit 24 Knoten.}
\label{TreppFunk1}
\end{table}

\begin{table}[htbp]
\centering
  \begin{tabular}{|c|c|}
  \hline
   Graphdistanz	& euklidische Distanz \\ \hline
		1	& $1$\\
		2	& $\sqrt{3}$, $2$\\
		3	& $\sqrt{7}$, $3$\\
		4	& $\sqrt{12}$, $\sqrt{13}$, $4$\\
		5	& $\sqrt{19}$, $\sqrt{21}$\\
		6	& $\sqrt{28}$\\
  \hline
  \end{tabular}
\caption{Graph- und euklidische Distanzen in einem Gridgraphen aus Dreiecken mit 23 Knoten.}
\label{TreppFunk2}
\end{table}

\begin{table}[htbp]
\centering
  \begin{tabular}{|c|c|}
  \hline
   Graphdistanz	& euklidische Distanz \\ \hline
		1	& $1$\\
		2	& $\sqrt{2}$, $2$\\
		3	& $\sqrt{5}$, $3$\\
		4	& $\sqrt{8}$, $\sqrt{10}$, $4$\\
		5	& $\sqrt{13}$, $\sqrt{17}$\\
		6	& $\sqrt{18}$, $\sqrt{20}$\\
		7	& $5$\\
		8	& $\sqrt{32}$\\
  \hline
  \end{tabular}
\caption{Graph- und euklidische Distanzen in einem Gridgraphen aus Vierecken mit 25 Knoten.}
\label{TreppFunk3}
\end{table}

%\newpage
\FloatBarrier 
% -------------------------------------------------------------------------------
\subsection{Verbessern der Laufzeit}
Wie wir gesehen haben, ist das Modell mit den reellen Zahlen viel langsamer als es mit den ganzen Zahlen. Da aber das erste nah an der Praxis liegt, ist es von Interesse, die Laufzeit des Models zu verbessern. 
Wir beschreiben weiter verschiede Ansätze, die diesem Ziel dienen sollten.

\begin{itemize}
\item Beschränkung der Konstanten $M$\\
	Um die Nichtkonvexität des Problems wegen der Ungleichung $\lvert c_i-c_j\rvert\ge d_{dist(i,j)}$ zu beheben, haben wir die Umschreibung mit dem "großen M" benutzt (Ungleichungen \ref{m23}, \ref{m24}). Die hohen Werte vom $M$ vergrößern die zulässige Menge vom Problem \ref{m2} und verlangsamen somit die Optimierung. 
	Aus der klassischen Formulierung folgt , dass es genügt, dass $M\ge d_1+\lambda$ ist \citep[siehe][]{HalaszSummary}. Durch die Abschätzungen für $\lambda$ können wir somit auch die Konstante $M$ abschätzen.
	Alternative können wir $M$ durch den Lösung $\lambda_Z$ des entsprechenden klassischen Problem beschränken, z.B. durch $ M\le 3\lambda_Z$. Die Ergebnisse der Berechnung mit dem so beschränkten Konstanten $M$ sind in der Tabelle \ref{Table:NewM} zusammengefasst.
	
	\begin{table}[htbp]
	\centering
	\begin{tabular}{|c|c|c|c|c|c|}
	\hline Lattice& $d(x)=3-x$(new)  & $d(x)=3-x$(old) && $d(x)=4-x$(new)  & $d(x)=4-x$(old)\\ \hline 
		Hexagonal	&  6.41699	& 6.41699 && 16.6243 & 16.6243 \\ 
			& 1.82 sec	& 3.42 sec && 6651.87 sec & 1838.73 sec \\ \hline
		Triangular	& 9.78029	&  9.78029 && 21.9068 & 21.9068 \\
			& 8.88 sec	& 57.01 sec && 59064.01 sec& 103898.60 sec \\ \hline
		Square	& 8.63494	&   8.63494 &&  19.9067 & 19.9067\\
			& 7.25 sec & 33.05 sec && 21222.8 sec & 27793.8 sec \\
	\hline
	\end{tabular}
	\caption{ Der Vergleich der Laufzeit des Modells und des Modells mit den zusaetzlichen Beschränkungen auf Konstanten $M$:
	für $d(x)=3-x$ ist $M\le 2\lambda_Z$, für $d(x)=4-x$ ist $M\le 3\lambda_Z$, wobei $\lambda_Z$  das Ergebnis des entsprechenden klassischen Problems ist, also $L(2,1)$ und $L(3,2,1)$.} 
	\label{Table:NewM}
	\end{table}
	
	Wie können sehen, dass bis auf einen Lauf ist das Modell schneller geworden. Die Erhöhung der Laufzeit im Fall des Graphen aus den Sechsecken kann daran liegen, dass die hinzugefügte Ungleichungen nicht hinreichend sind. Sie schaffen zusätzliche Schwierigkeiten für das Programm, da sie erfüllt werden müssen, helfen aber beim Branch-and-Bound nicht. 
\item Beschränkung von $\lambda$ 

	Ein anderer Ansatz ist zu versuchen, gute obere Schranke für $\lambda$ zu finden. Dafür haben wir den Algorithmus der Binäresuche verwendet.
	
	Die Idee ist, solange die obere und untere Schranken vom $\lambda$ zusammen zu ziehen, bis die Lücke zwischen ihnen klein genug ist, und erst danach die Optimierung mit den strengeren Schranken zu starten.
	Als untere Anfangsschranke für $\lambda$ wurde das Ergebnis der entsprechenden klassischen Formulierung genommen ($\lambda_Z$), und als obere $2\lambda_Z$ für die Funktion $d(x)=3-x$ und $3\lambda_Z$ für die Funktion $d(x)=4-x$. In jeder Iteration wird versucht entweder untere Schranke nach oben um die Hälfte der Lückenlenge, oder obere nach unten zu verschieben. Dabei wird jedes mal ein Zulässigkeitsproblem gelöst. 
	
	Die Ergebnisse dieses Ansatzes für die zulässige Lückengröße $0.5$ sind in der Tabelle \ref{Table:BB} aufgeführt.
	
	\begin{table}[htbp]
	\centering
	\begin{tabular}{|c|c|c|c|c|c|}
	\hline Lattice& $d(x)=3-x$(new)  & $d(x)=3-x$(old) && $d(x)=4-x$(new)  & $d(x)=4-x$(old)\\ \hline 
		Hexagonal	&  6.41699	& 6.41699 && 16.6243 & 16.6243 \\ 
			& $26.77+21.59$ sec	& 3.42 sec && $257637+139576$ sec & 1838.73 sec \\ \hline
		Triangular	& 9.78029	&  9.78029 && 21.9068 & 21.9068 \\
			& $89.25+55.26$ sec	& 57.01 sec && & 103898.60 sec \\ \hline
		Square	& 8.63494	&   8.63494 &&  19.9067 & 19.9067\\
			& $358.97+253.17$ sec & 33.05 sec &&  & 27793.8 sec \\
	\hline
	\end{tabular}
	\caption{ {\it Branch-and-Bound} Ansatz für die Bestimmung oberer und unterer Schranke für $\lambda$. Die Lückengröße ist $0.5$. Der erste Summand in der Summe steht für die Zeit, die die Suche nach den besseren Schranken braucht, und der zweite Summand steht für die eigentliche Optimierungszeit.} 
	\label{Table:BB}
	\end{table}
	
	Man kann sehen, dass dieser Verfahren im Gegensatz zu den Erwartungen keine Zeitverbesserung geleistet hat. Es liegt laut (REFERENZ) daran, dass die Aufgabe die Zulässigkeit vom $CAP$ zu prüfen, mindestens $NP-$schwer ist.
	
	Außerdem, nach dem wir die zulässige Menge verkleinert haben, ist es schwer im Suchbaum einige Zweige abzuschneiden (Bounding) und die Suche nach dem Optimum dauert länger als davor.
	
	Eine andere Möglichkeit $\lambda$ abzuschätzen ist die Untersuchung von den Teilgraphen des ursprünglichen Graphen $G$. Z.B. die Lösung $\lambda'$ des Problems für den Teilgraphen $G'$, die das Minimum der Summe $\sum_{i\in V(G')}{c(i)}$ gewährleistet, liefert die untere Schranke für die Lösung $\lambda$ des ursprünglichen Graphen $G$.
	
	Die Form und die Größe der Teilgraphen $G'$ kann variiert werden. Wir haben hauptsächlich die Bauelemente der Gridgraphen (entspricht Dreiecke, Vierecke und Sechsecke) in Betracht benommen, u.a.: Dreiecke mit der Seitenlänge 1, 2, 3, Sechsecke mit der Seitenlänge 1,2 , Trapez, das die Hälfte eines Sechseckes darstellt, Vierecke der Größe 1, 2, 3 und kleine viereckige Gridgraphen mit 4, 9 Knoten. 
	
	Für die Teilgraphen löst man separat das Problem der Minimierung der Summe der Labels, die die Ungleichungen \ref{m21}-\ref{m26} erfüllen. Für den ganzen Gridgraphen fügt man für alle Knoten, die die bestimmten Teilgraphenstruktur bilden, zusätzliche Ungleichungen hinzu.
	
	Die Zusammenfassung der besten Ergebnisse fuer verschiedene Grridgraphen und Teilgraphen findet man in den Tabellen \ref{Table:TG1}, \ref{Table:TG2}, \ref{Table:TG3}.

	\begin{table}[htbp]
	\centering
	\begin{tabular}{|c|c|c|}
	\hline Hexagonal Lattice& $d(x)=3-x$  & $d(x)=4-x$\\ \hline 
		Der ganze Graph	&  6.41699	& 16.6243 \\ 
			& $3.42$ sec	& $1838.73$ sec \\ \hline
		Teilgraphen & & \\\hline
		Sechsecke mit der Seitenlänge $1$& $1.37$ sec	&  $1588.7$ sec \\ \hline
		Sechsecke mit der Seitenlänge $1$, $2$	& $1.75$ sec	&  $1641.71$ sec \\ \hline
		Trapez& $1.77$ sec	&  $1529.24$ sec \\ \hline
	\end{tabular}
	\caption{Untersuchung der Teilgraphen, Fall der Gridgraphen aus der Sechsecken.} 
	\label{Table:TG1}
	\end{table}
	
	\begin{table}[htbp]
	\centering
	\begin{tabular}{|c|c|c|}
		\hline Triangular Lattice& $d(x)=3-x$  & $d(x)=4-x$\\ \hline 
		Der ganze Graph&  $9.78029$	& $21.9068$ \\ 
			& $10.96$ sec	& $103883$ sec \\ \hline
		Teilgraphen & & \\\hline
		Dreiecke mit der Seitenlänge $1$& $12.59$ sec	& $46933$ sec \\ \hline
		Dreiecke mit der Seitenlänge $1$,$2$,$3$& $10.02$ sec	& $92228.1$ sec \\ \hline
		Sechsecke mit der Seitenlänge $1$ &  $11.90$sec	&  $42412$sec \\ \hline
	\end{tabular}
	\caption{Untersuchung der Teilgraphen, Fall der Gridgraphen aus der Dreiecken.}
	\label{Table:TG2}
	\end{table}
	
	
	\begin{table}[htbp]
	\centering
	\begin{tabular}{|c|c|c|}
	\hline Square Lattice& $d(x)=3-x$  & $d(x)=4-x$\\ \hline 
		Der ganze Graph	&  8.63494	& 19.9067 \\ 
			& $4.92$ sec	& $27793.8$ sec \\ \hline
		Teilgraphen & & \\ \hline
		Vierecke mit der Seitenlänge $1$& $12.76$ sec	& $37945.53$ sec \\ \hline
		Vierecke mit der Seitenlänge $1$,$2$,$3$ ohne Überlappen& $7.11$ sec	& $19619.38$ sec \\ \hline
		Vierecke $1\times 2$ & $6.22$ sec	& $28612.78$ sec \\ \hline
	\end{tabular}
	\caption{Untersuchung der Teilgraphen, Fall der Gridgraphen aus der Vierecken.} 
	\label{Table:TG3}
	\end{table}

	Bei der Gridgraphen, die aus der Sechsecken bestehen, liefert der beschriebene Ansatz keine Verbesserung der Laufzeit.
	
	Bei der Gridgraphen aus der Dreiecken kann man eine Verbesserung der Laufzeit beobachten. Aber immerhin braucht die Berechnung viel Zeit und mit der Einführung der zusätzlichen Nebenbedingungen auch mehr Speicherplatz. 
	
	Und im dritten Fall des Gitters aus den Vierecken hat der Ansatz wiederum keine deutliche Verbesserung der Laufzeit gebracht.

\end{itemize}
\FloatBarrier 
% Literaturverzeichnis
\newpage
\addcontentsline{toc}{section}{Literatur}

\bibliographystyle{plainnat}
\bibliography{literatur}


\end{document}